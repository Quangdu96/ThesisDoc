\documentclass[class=report, crop=false]{standalone}
\usepackage{../preamble}

\begin{document}
	% Specific lengths in this document
	\import{../}{default_lengths.tex}
	%
	\chapter{Kết luận đề tài và hướng phát triển}\label{chap:7}
	\section{Kết quả đạt được}
		Qua đề tài này, sinh viên học được rất nhiều kiến thức của lý thuyết mật mã, toán~học và khoa học máy tính, từ đó hiểu rõ lĩnh vực nghiên cứu mật mã, đặc biệt là mã hóa dựa trên định danh, từ bức tranh tổng quát đến chi tiết, kỹ thuật, phương pháp cụ thể, tạo tiền đề tốt để tiếp tục nghiên cứu sau này.

		Về khía cạnh ứng dụng, sinh viên đã học được nhiều góc nhìn sâu sắc về bối cảnh áp dụng thực tế, các vấn đề nảy sinh, phương pháp giải quyết, cùng với đó là tìm hiểu các tiêu chuẩn công nghiệp.
		
		Quan trọng hơn hết, sinh viên đã đưa được hệ mã BBG từ một công trình lý thuyết qua các cải tiến đến được cài đặt. Hệ mã được cài đặt có độ hiệu quả tốt ở cả thời gian và không gian.
	%
	\section{Hạn chế và thiếu sót}
		Khóa luận này không được đầy đủ như sinh viên dự tính. Những kiến thức về đường cong elliptic và cặp ghép, vấn đề thám mã và chọn tham số cho nhóm song tuyến tính, dù đã được tìm hiểu qua nhưng đã không được trình bày. Dù vậy sinh viên có thể đảm bảo rằng phần lớn khóa luận là có thể hiểu được mà không cần những kiến thức trên. \\

		Ngoài ra, sinh viên nhận định rằng nếu muốn phát triển hệ mã BBG thành thư viện cho cộng đồng, cần viết lại hệ mã bằng ngôn ngữ lập trình bậc cao như Python để tăng tính thân thiện. Tuy nhiên chỉ áp dụng công nghiệp được với điều kiện là có chuẩn cho hệ mã BBG, một điều xem ra là khó có được trong tương lai gần.
	%
	\section{Hướng phát triển của đề tài}
		Đề tài này có thể được nối tiếp bằng việc nghiên cứu mở rộng sang các lược đồ cao cấp khác liên quan như mã hóa dựa trên thuộc tính, mã hóa tìm kiếm được, mã hóa an toàn trong tương lai v.v. Hoặc như đã nói ở chương \ref{chap:2}, đó là hoàn thiện IBE thành một con ``dao Thụy Sỹ''. Tuy nhiên xu hướng hiện giờ là xây dựng lại IBE trên lý thuyết~dàn. Do trên dàn các lược đồ cơ bản như mật mã khóa công khai và chữ ký vẫn chưa hoàn thiện nên nghiên cứu IBE ở đây còn rất sơ khởi so với trong cặp~ghép. Một~hướng nghiên cứu gần với ứng dụng hơn là tìm cách \emph{phi tập trung hóa} (decentralize), loại bỏ thực thể trung tâm PKG trong IBE.
	\newpage
	%
	% Reset default lengths
	\import{../}{default_lengths.tex}
\end{document}
