\documentclass[class=report, crop=false]{standalone}
\usepackage{../preamble}

\begin{document}
	% Specific lengths in this document
	\import{../}{default_lengths.tex}
	%
	\chapter{Mở đầu}
	\section{Giới thiệu khóa luận}
		% Đề xuất của Shamir + Lý do chọn đề tài
		Sau khi mã hóa khóa công khai được đề xuất, bài toán trao đổi khóa của mã hóa khóa bí mật được giải quyết. Tuy nhiên từ đó lại nảy sinh vấn đề mới cũng trong trao đổi khóa là xác thực chủ sở hữu khóa công khai. Để giải quyết vấn đề này, một số giải pháp đã được đưa ra, điển hình là cơ sở hạ tầng khóa công khai (public key infrastructure, từ đây gọi tắt là PKI). Dựa vào phương pháp chứng nhận khóa công khai, PKI còn được chia làm hai nhánh là PKI dựa trên CA (certificate authority) hoặc dựa trên mạng lưới tin tưởng (web of trust). Nếu xét về các kỹ thuật mật mã được sử dụng thì PKI dù là hình thức nào cũng chỉ sử dụng những mật mã nguyên thủy (cryptographic primitive) là hàm băm, mã xác thực thông điệp (message authentication code), mã hóa khóa bí mật, mã hóa khóa công khai và chữ ký điện tử. Đây được xem là những mật mã nguyên thủy cấp thấp.

		Năm 1984, Shamir \cite{DBLP:conf/crypto/Shamir84} đưa ra khái niệm \textit{mã hóa dựa trên định danh} (identity-based encryption, từ đây gọi tắt là IBE) nhằm giải quyết vấn đề trên theo tiếp cận xây dựng một lược đồ mật mã cao cấp, tức là có thêm chức năng đặc biệt. Tuy nhiên lúc đó Shamir không đưa ra được một hệ mã cụ thể. Mãi đến năm 2001, hai hệ IBE đầu tiên mới được công bố bởi Boneh, Franklin \cite{DBLP:conf/crypto/BonehF01} và Cocks \cite{DBLP:conf/ima/Cocks01}. Từ đó đến nay rất nhiều hệ IBE khác đã được công bố và lịch sử đã chứng minh ý nghĩa to lớn của IBE trong cả lý thuyết mật mã lẫn ứng dụng. Trong lý thuyết, IBE là nguyên liệu để xây dựng các mật mã nguyên thủy cao cấp khác. Còn ở khía cạnh ứng dụng, việc mã hóa sử dụng định danh (là một chuỗi ký tự có ý nghĩa) mở ra nhiều kỹ thuật thiết kế các hệ thống an toàn và dễ quản lý. Vì những lý do trên, sinh viên quyết định chọn IBE làm đề tài nghiên cứu của khóa luận này.
	%
	\section{Đối tượng nghiên cứu}
		\begin{itemize}
			\item Về lý thuyết, sinh viên chọn nghiên cứu các hệ IBE và các lược đồ mật mã liên quan. Tất cả các lược đồ này đều dựa trên cơ sở toán là cặp ghép trong lý thuyết đường cong elliptic.
			\item Về ứng dụng, sinh viên nghiên cứu các mô hình ứng dụng thực tế của IBE và các tiêu chuẩn quốc tế hiện có.
		\end{itemize}
	%
	\section{Mục tiêu khóa luận}
		\begin{itemize}
			\item Nghiên cứu lý thuyết đường cong elliptic và cặp ghép để hiểu chúng được ứng dụng vào mật mã như thế nào và biết cách áp dụng hợp lý để hệ mã an toàn mà vẫn hiệu quả.
			\item Học được phương pháp nghiên cứu mật mã lý thuyết và chứng minh chặt chẽ độ an toàn của một hệ mã.
			\item Hiểu được những vấn đề được quan tâm liên quan đến IBE trong lý thuyết và thực tiễn.
			\item Tìm hiểu các ứng dụng của IBE.
			% \item Cài đặt một hệ IBE tối ưu và đánh giá độ hiệu quả thực tế.
		\end{itemize}
	%
	\section{Phạm vi khóa luận}
		Trong khóa luận này, sinh viên xin lần lượt trình bày những kiến thức nền tảng về IBE và những khía cạnh xung quanh; sau đó là mô tả chi tiết hệ mã Boneh-Boyen-Goh, cùng việc kết hợp các cải tiến sẵn có để có thể hiện thực hệ mã bằng phần mềm; cuối cùng nói về một số ứng dụng của cặp ghép và IBE.

	\newpage
	%
	% Reset default lengths
	\import{../}{default_lengths.tex}
\end{document}
