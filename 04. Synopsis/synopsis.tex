\documentclass[class=report, crop=false]{standalone}
\usepackage{../preamble}

\begin{document}
	% Specific lengths in this document
	\import{../}{default_lengths.tex}
	\tabulinesep=0.5\baselineskip
	%
	\hypertarget{synopsis}{}
	\bookmark[dest=synopsis,level=0]{Đề cương chi tiết}
	\thispagestyle{empty}
	\rmfamily
	\noindent
	Khoa Công Nghệ Thông Tin \\
	Bộ môn Công Nghệ Tri Thức
	%
	\begin{center}
		\Large\bfseries
		ĐỀ CƯƠNG CHI TIẾT
	\end{center}
	%
	\normalsize
	\begin{longtabu}{| X[c] | X[c] |}
		\tabucline[1pt]-
		\everyrow{\tabucline[1pt]-}
		\multicolumn{2}{| p{\textwidth} |}{
			{\bfseries Tên đề tài:} Tìm hiểu mã hóa dựa trên định danh và ứng dụng
		} \\
		\multicolumn{2}{| p{\textwidth} |}{
			{\bfseries Giáo viên hướng dẫn:} TS. Lê Văn Luyện
		} \\
		\multicolumn{2}{| p{\textwidth} |}{
			{\bfseries Thời gian thực hiện:} Từ 01/01/2019 đến 20/06/2019
		} \\
		\multicolumn{2}{| p{\textwidth} |}{
			{\bfseries Sinh viên thực hiện:}  Huỳnh Quang Dự -- 1412114
		} \\
		\multicolumn{2}{| p{\textwidth} |}{
			{\bfseries Loại đề tài:} Tìm hiểu lý thuyết và cài đặt thử nghiệm
		} \\
		\multicolumn{2}{| p{\textwidth} |}{
			{\bfseries Nội dung đề tài:} \newline
			Về lý thuyết:
			\begin{itemize}
				\item Phân tích mã hóa dựa trên định danh và những yếu tố xung quanh được quan tâm, trong đó có các tính năng, khái niệm an toàn, các bài toán khó, vấn đề và tiêu chí đánh giá một hệ mã
				\item So sánh chi tiết trên nhiều khía cạnh các hệ mã đã có
				\item Áp dụng các cải tiến sẵn có để xây dựng một hệ mã hoàn chỉnh, an toàn, tối ưu, tính năng tốt và cài đặt được
				\item Tìm hiểu các lược đồ mật mã khác có thể xây dựng từ cặp ghép song tuyến tính và mã hóa dựa trên định danh
			\end{itemize}
			Về ứng dụng:
			\begin{itemize}
				\item Cài đặt thử nghiệm hệ mã Boneh-Boyen-Goh bằng thư viện PBC trên C++
				\item Trình bày một số ứng dụng thực tiễn của mã hóa dựa trên định danh
				\item Tìm hiểu tiêu chuẩn về mã hóa dựa trên định danh
			\end{itemize}
			%
		} \\
		\hline
		\multicolumn{2}{| p{\textwidth} |}{
			{\bfseries Kế hoạch thực hiện:}
			\begin{itemize}
				\item Từ 01/01/2019 đến 31/01/2019
				\begin{itemize}
					\item[$\diamond$] Nhận đề tài
					\item[$\diamond$] Đọc các bài báo khảo sát
					\item[$\diamond$] Quyết định nền tảng toán để tìm hiểu mã hóa dựa trên định danh
				\end{itemize}
				%
				\item Từ 01/02/2019 đến 31/03/2019
				\begin{itemize}
					\item[$\diamond$] Tìm hiểu lý thuyết đường cong elliptic và cặp ghép song tuyến tính
					\item[$\diamond$] Đọc các bài báo của các công trình quan trọng trong lĩnh vực mã hóa dựa trên định danh
					\item[$\diamond$] So sánh và chọn ra hệ mã thích hợp để nghiên cứu và cài đặt
					\item[$\diamond$] Tìm hiểu bối cảnh ứng dụng được mã hóa dựa trên định danh và các tiêu chuẩn hiện hành
				\end{itemize}
				%
				\item Từ 01/04/2019 đến 30/04/2019
				\begin{itemize}
					\item[$\diamond$] Soạn thảo dàn ý chi tiết và viết những nội dung đã tìm hiểu được
					\item[$\diamond$] Áp dụng các kỹ thuật cải tiến để xây dựng hệ mã hoàn thiện có thể cài đặt
					\item[$\diamond$] Tìm kiếm thư viện hỗ trợ và quyết định nền tảng cài đặt
				\end{itemize}
				%
				\item Từ 01/05/2019 đến 20/06/2019
				\begin{itemize}
					\item[$\diamond$] Hoàn chỉnh khóa luận
					\item[$\diamond$] Cài đặt và đánh giá độ hiệu quả
				\end{itemize}
			\end{itemize}
		} \\
		\everyrow{}
		Xác nhận của GVHD & Ngày \makebox[1cm]{\dotfill} tháng \makebox[1cm]{\dotfill} năm \makebox[1cm]{\dotfill} \newline
		Sinh viên thực hiện \\
		& \\
		& \\
		& \\
		TS. Lê Văn Luyện & Huỳnh Quang Dự \\
		\tabucline[1pt]-
	\end{longtabu}
	\newpage
	%
	% Reset default lengths
	\import{../}{default_lengths.tex}
\end{document}
