\documentclass[class=report, crop=false]{standalone}
\usepackage{../preamble}


\begin{document}
	% Specific lengths in this document
	\import{../}{default_lengths.tex}
	%
	\chapter*{Tóm tắt khóa luận}
	\addcontentsline{toc}{chapter}{Tóm tắt khóa luận}
	%
	Khóa luận này có nội dung chủ yếu về mã hóa dựa trên định danh, với nền tảng toán là cặp ghép song tuyến tính trong lý thuyết đường cong elliptic. Chương mở đầu nói về đối tượng nghiên cứu, mục tiêu và phạm vi của khóa luận.

	Ở đầu chương 2, sinh viên xin trình bày nguyên nhân khởi nguồn cho ý tưởng mã hóa dựa trên định danh, kết hợp với mô tả giao thức vận hành một cách trực quan và so sánh với cơ sở hạ tầng khóa công khai (PKI). Kế đến là phần bàn luận về các tính năng và vấn đề trong mã hóa dựa trên định danh, có cả khía cạnh kỹ thuật lẫn xã hội. Tiếp theo là định nghĩa hình thức của hệ mã, khái niệm an toàn trong đó và các bài toán khó, tất cả được trình bày hình thức theo như lý thuyết mật mã. Chương 2 được kết lại với bàn luận về cách đánh giá một hệ mã hóa dựa trên định danh cùng với khảo sát về các công trình nổi bật.

	Chương 3 được dành riêng cho việc trình bày hệ mã Boneh-Boyen-Goh. Qua các mục trong chương là từng bước nâng cấp hệ mã để đạt được phiên bản tối ưu cài đặt được. Mỗi nâng cấp được trình bày rất chi tiết. Kèm theo đó là so sánh độ hiệu quả thời gian và không gian với một vài hệ mã nổi tiếng khác. Cuối chương là kết quả thử nghiệm thực tế được sinh viên chạy thử trên máy tính cá nhân.

	Một số ứng dụng của cặp ghép và mã hóa dựa trên định danh sẽ được trình bày trong chương 4. Về mặt lý thuyết, sinh viên giới thiệu một số lược đồ mật mã khác được xây dựng từ cặp ghép và mã hóa dựa trên định danh. Phía ngược lại, phục vụ cho mục đích ứng dụng thực tiễn của khóa luận này, một số bối cảnh xã hội có thể triển khai mã hóa dựa trên định danh sẽ được bàn luận cụ thể.

	Chương 5 là kết luận của đề tài này.
	\newpage
	%
	% Reset default lengths
	\import{../}{default_lengths.tex}
\end{document}
